\documentclass[12pt, a4paper]{ctexbook} 
% \documentclass[12pt, a4paper]{article} %book,article 为书、文章等等,写作的不同类型
\usepackage[utf8]{inputenc}
\usepackage{ctex} 
\usepackage{graphicx}
\usepackage{amsmath}
\usepackage{bm}
\usepackage{indentfirst}
\usepackage{makecell, rotating, multirow, diagbox}
\usepackage{longtable}
\usepackage{booktabs}%表格设置
\usepackage[colorlinks=true,linkcolor=red,urlcolor=blue,anchorcolor=blue,citecolor=blue]{hyperref}
\usepackage{ulem}
\usepackage{color}

%%
\usepackage{geometry}
\geometry{top=3cm,bottom=3cm}
% \geometry{left=2.5cm,right=2.5cm,top=2.5cm,bottom=2.5cm}
\usepackage{fancyhdr}
\footskip = 10pt
\pagestyle{fancy}
\lhead[]{页眉} % lead[偶数页眉]{奇数页眉},只能选择偶奇中的一种,A为你想要填写的内容
\rfoot[偶数页脚]{}
\chead{昕}% \rhead --> 右页眉   \lfoot --> 左页脚  \rfoot --> 右页脚 \chead --> 页眉中间 \cfoot --> 页脚中间
\renewcommand{\headrulewidth}{0.4pt}
\renewcommand{\footrulewidth}{0.4pt}
%% 自定义页边距、用纸类型等等。自定义页眉、页脚等,这里是整体篇幅定义页眉页脚设置

%%
\setlength{\parindent}{2em}                 % 首行两个汉字的缩进量
\setlength{\parskip}{3pt plus1pt minus1pt} % 段落之间的竖直距离 
% \setlength{\voffset}{-10mm}                        
% \setlength{\topmargin}{0mm}
% \setlength{\headheight}{5mm}
% \setlength{\headsep}{5mm}
\setlength{\footskip}{10mm}
\renewcommand{\baselinestretch}{1.5}  % 定义行距
%% 此部分为文字间距设置,里面具体的数字可以修改到合适值,看你准备投放的期刊、书籍的要求是多少。所以开始书写正文时 这里的值可以先不用考虑修改

%%
\CTEXsetup[beforeskip={0pt}]{chapter}%这里设置的是章标题与上页边距的距离,默认时是比较大的,记得自己设置
\CTEXsetup[name = {第,篇},number = {\chinese{chapter}},format = {\textbf\heiti\zihao{1}\centering}]{chapter} 
\CTEXsetup[name = {第,章},number = {\chinese{section}},format = {\textbf\heiti\zihao{2}\leftline}]{section} 
\CTEXsetup[name = {},number = {\arabic{subsection}},format = {\textbf\songti\zihao{3}\leftline}]{subsection}
\CTEXsetup[name = {(,)},number = {\arabic{subsubsection}},format = {\textbf\songti\zihao{3}\leftline}]{subsubsection} 
% name:章节名称,
% number:序号形式,包括以下这些-->
% \arabic (1, 2, 3, ...)
% \alph (a, b, c, ...)
% \Alph (A, B, C, ...)
% \roman (i, ii, iii, ...)
% \Roman (I, II, III, ...)
% \chinese(一,二,三,...)
% \fnsymbol (?, ?, ?, §, ?, ...)
% 不加修饰表默认形式(1,1.1,1.1.1)
% format:字体样式 
%% 此部分为中文图书标题、目录等的设置


\title{\textbf{环境诉讼}}
\author{王怡昕博士 \quad 上海财经大学}
% \address{上海市}
\date{\today}


\begin{document}

%%
\maketitle
\thispagestyle{empty}

\frontmatter
\chapter{内容简介}
本书主要是讲。。。。

\newpage

\tableofcontents
\newpage
%%文章内容书写前的目录准备

%%
\pagestyle{fancy}
\fancyhead{} % 重置页眉页脚
\fancyhead[LO,RO]{\CJKfamily{hei}\bfseries\zihao{5} 这里写啥1}
\fancyhead[LE,RE]{\CJKfamily{hei}\bfseries\zihao{5} 这里写啥2}
\fancyfoot[LO]{左奇数页脚注}
\fancyfoot[LE]{左偶数页脚注}
\fancyfoot[RO]{右奇脚注}
\fancyfoot[RE]{右偶脚注}
%% 这里的页眉页脚设置指此处以下的文档内容,在这里之前的文档页眉页脚仍然是前面的设置

\mainmatter

\chapter{前言}
本书主要研究环境公诉和私诉两种诉讼类型。。。
\section{现状}
现状
\section{分析}
分析

\begin{figure}
    \centering
    \includegraphics[width=0.5\textwidth]{figures/boy.jpg}
    \caption{\label{boy}哈哈。大概就这些内容了。\LARGE\textcolor{red}{有其他新的想法了我来给你扔进书里面,可以先看看这些加在书里面后,书的整体的感觉。}}

\end{figure}
选题依据  

Nb3Sn超导线材自从发现以来,已经经历了多代工艺的发展,主要可以分为粉末装管法(Powder in Tube Process,PIT)、青铜法(Bronzen Process)和内锡法(Internal Tin Process,IT)等考虑到加工工艺可行性以及载流能力等因素,目前工业商业应用上最为广泛的工艺技术主要为青铜法和IT工艺,但对于具有很高磁场和临界载流能力(通常4.2K,大于10T)的应用场景,比如ITER项目,CFETR项目,高磁场基础科研大装置等,虽然高温超导线材的载流能力在很高磁场需求下可以实现,比如钇系和铋系高温超导线材,但高温线材的工业化制备面临的工艺复杂性、成本和大规模生产等还不能跟低温线材形成竞争局面。因此,面对不同的高磁场需求,目前能够保证其需求的工艺路线基本全部采用内锡法工艺,全球范围内也催生了几条主要的内锡法超导线材制备路线,比如Ta单阻隔层内锡法、改进型Jelly Roll工艺(MJR),分布式Nb阻隔层高Jc 线材工艺等等。相比之下,单阻隔层内锡设计和分布式阻隔层设计的线材在工业化大批量制备路线上具有很大的优势,其结构线材的易加工性和性能大范围的可调节性,使得这两种技术路线成为了当下高磁场应用最为广泛的选择。目前,采用分布式阻隔层设计线材实现了最高性能,已经达到3000A/mm2(4.2K,12T,美国OST公司拥有这种最高性能线材的知识产权)。虽然国内对分布式阻隔层Nb3Sn超导线材研究和工业化制备也来到了2800A/mm2 (4.2K,12T)的水平,但这种高Jc线材的长线加工工艺仍不稳定,其长线的获得仍然是进一步的深入研究探索的主要方向之一。另一方面,分布式阻隔层设计的Nb3Sn线材仍处于国外技术专利阶段,为了更好的形成产品竞争力和超导产品选择,对于专利技术的突破已经迫在眉睫,如何突破分布式阻隔层设计,且同时获得具有较高载流能力的新型Nb3Sn结构线材,也已经成为了未来超导产品开发必须回答和选择的必经之路。因此,设计新型高Jc Nb3Sn超导线材,不仅可以更加深刻的认识到高性能Nb3Sn超导线材的关键设计参数,为进一步提高超导线材性能提供可行的设计思路和方案,使国内在高性能超导线材领域具有更深的技术积累。同时,新型结构的高性能Nb3Sn超导线材将具有完全自主的研发技术和方案,进一步提升在超导线材领域的国际竞争力。而对新型结构线材长线加工工艺探究,可以为不同型号超导线材的长线加工研究提供一定的普适研究方法,进一步夯实国内在超导线材长线加工工艺方面的技术深度。

研究内容(研究对象 拟解决的关键科学问题 研究目标)2000
研究对象:
新型高性能 Nb3Sn超导线材结构设计,和与新结构对应的长线加工研究。

拟解决的关键科学问题:
1 新结构Nb3Sn超导线材的高性能来源,获得非分布式阻隔层的高性能Nb3Sn超导线材。Nb3Sn超导线材性能包括承载临界电流Ic、电流密度Jc、剩余电阻率RRR、磁滞损耗Qh和n值等,这些参数主要跟线材中元素原子比例、元素间距、线材热处理制度、线材尺寸等相关,因此要获得高性能Nb3Sn线材就需要从这些参数方面出发。本项目新结构Nb3Sn线材研究的主要出发点在元素比例、元素间距和热处理制度等方面,主要从这三个方面来实现新结构的设计。需要注意的是 ,同一种结构线材在不同线径下晶粒特征在热处理后会有所区别,本项目暂不考虑与线径的关系,新结构设计时采用相同线径。因此,本项目研究性能相关的拟解决的关键科学问题有:
1) 不同原子比例的Cu、Nb、Sn等元素对热处理生成Nb3Sn晶粒结构的影响,比如Nb3Sn晶粒的大小和分布等,并获得特定新型高性能Nb3Sn结构线材与元素比例的关系。Cu-Sn、Nb-Sn等元素原子比与最终热处理后的Nb3Sn线材性能紧密相关,在不同的原子比下会形成多种复杂的合金结构,这些复杂的结构会改变元素的扩散效率,从而影响最终Nb3Sn的生成。另外,高Jc Nb3Sn线材的研究结果表明,Sn不足导致没有的Sn来生成Nb3Sn,Sn过量则可能导致过多的Sn生成其它合金相从而降低Nb3Sn线材的整体性能,而Nb的量直接决定了生成Nb3Sn晶粒量的多少,Cu不足可能导致Nb3Sn大晶粒过多或者Sn元素扩散通道变窄,Cu的过量则可能会消耗更多的Sn而间接导致Nb3Sn生成量减少等等。因此,新结构线材需合理选择元素比例,尽量获得具有类似高性能线材的Nb3Sn晶粒特征,比如分布式阻隔层结构高Jc Nb3Sn线材的晶粒大小和分布,以此来保证新结构线材的性能。 
2) 不同Cu、Nb、Sn等元素的相对间距对Nb3Sn超导线材性能的影响。在不同热处理阶段Cu、Nb、Sn等元素的扩散速度会有所不同,元素间距的不同会形成不同的元素扩散梯度,从而在不同位置形成不同的元素原子比,影响元素间最终能够形成合金结构。因此,元素间距的合理选择可能会影响Nb3Sn线材热处理时的各个阶段,这也是新结构线材设计时必须考虑和解决的问题,具体包括Nb芯丝间距、Nb-Sn元素间距等。
3) 新结构Nb3Sn线材热处理制度的研究,以及影响热处理制度的关键参数探究。为了使由纯金属元素以特定结构构成的线材能够在热处理制度下生成具有高载流能力的Nb3Sn超导晶粒,不同结构设计的Nb3Sn超导线材对应着不同的元素原子比、扩散间距等,其热处理过程也需要根据结构的改变在原有热处理制度上做出调整。而为了能获得更为有效和合理的调整方案,对影响热处理的关键参数的探究具有更现实和迫切的需要,比如元素扩散通道的宽度、距离、原子比、晶粒大小和生长、芯丝搭接范围等等,这些参数研究也是本项目研究的一个重点方向。
2 新结构Nb3Sn超导线材长线加工工艺的研究。目前采用分布式阻隔层结构的高性能Nb3Sn线材通过挤压、拉伸等工艺,使线材加工成型时其内部Nb芯丝的尺寸在1 μm ~ 3μm之间,亚组元尺寸在50μm ~ 80μm之间(跟线材直径相关)。这种微米尺度的金属线材极限拉拔加工过程本身具有很大的挑战性,要获得百米级甚至千米级的Nb3Sn长线,根据本单位对高Jc Nb3Sn线材长线的研究基础,主要从加工工艺、原材料、结构设计等方面考虑,这些方面所涉及的具体研究内容包括必须严格、准确地控制线材在各个工艺阶段的加工质量,控制Nb芯丝在长线加工时变形在合理的范围内并且不断芯丝,以及控制住线材中亚组元在微米级别的变形和完整性等。因此,对于长线加工工艺的研究,根据之前的研究部基础,本项目主要需要解决以下几个问题:
1) 影响长线加工关键工艺研究。主要包括线材拉伸速率、加工率等,保证加工时线材结构中的各个构成部件能够较协同一致的产生变形。
2) 亚组元变形研究。结构设计的不同,会使加工时线材受力分布产生改变,这种受力的改变会使亚组元尺度的结构产生不均匀的变形。为了保证亚组元在不均匀受力时内部结构不受到破坏,合理的结构设计具有一定的研究价值,主要包括亚组元结构对称性设计、异型亚组元长线加工变形及可行性、亚组元中Cu元素比例控制(Cu元素增加可以增加长线可加工性)等。
3) Nb芯丝变形及芯丝断线研究。本单位的研究发现,高Jc Nb3Sn 长线获得目前最主要的困难在于控制Nb芯丝单根芯丝的断线和变形,因此在新结构Nb3Sn线材中如何控制单根Nb芯丝的断线问题和变形问题也将是主要的研究重点。

研究目标:
1设计新型结构的高性能Nb3Sn的超导线材,获得性能超过2300A/mm2(4.2K,12T),且具有自主知识产权的Nb3Sn超导线材,并获得性能相关的Nb3Sn线材结构设计的关键设计参数信息。
2研究新型结构Nb3Sn超导线材的长线加工工艺关键参数信息,能够制备出线材长度 > 500m的Nb3Sn超导线材。


研究方案2000
新型高性能Nb3Sn超导线材设计研究设计原则:
1) 避免采用分布式阻隔层Nb3Sn结构。
2) Nb3Sn线材中Nb元素含量决定线材性能上限,设计接近高Jc线材的元素比例。
3) 新结构的获得尽量在当前线材加工工艺下能实现。

方案设计关键参数:
1) 阻隔层分布。阻隔层在Nb3Sn线材中的作用主要在于控制Cu、Sn等元素的扩散距离,以此来保证线材的性能,新结构中亚组元Nb阻隔层形式的结构设计必须避免,可选的研究方案有:a) 内锡法线材亚组元无阻隔层、复合线单Ta阻隔层设计,b) 无阻隔层线材设计, 或者c) 分布式阻隔层变形结构设计。根据本单位实际情况,本项目主要集中在a方案阻隔层分布形式的新结构设计研究上。
2) Nb与Sn原子比ε。Nb3Sn中接近3:1的Nb/Sn原子比根据具体结构来选择大于或者小于这个原子比,比如高Jc Nb3Sn中比值一般为3.3~3.6之间,而内锡法线材中比值通常接近或者小于2.8。
3) Cu区与非Cu超导区体积比CuSc。主要是包括考虑相同Jc下线材Ic性能的问题,线材可加工性问题等。
4) Nb芯丝尺寸Φ。直接与线材性能相关,且不同Nb芯丝尺寸对应的热处理过程不同,通常需要保证Nb芯丝直径大于1μm。
5) Nb芯丝间距Δ。芯丝间距对影响Cu、Sn等元素的扩散效率,影响最终热处理Nb3Sn晶粒质量。
6) Nb3Sn晶粒大小。主要与大晶粒Nb3Sn对线材Ic、Jc等性能的影响相关。
7) 亚组元尺寸φ。对继续采用亚组元结构的新结构线材,需考虑最终线材中的亚组元尺寸,其对应着一定的Nb芯丝大小和芯丝间距。

单Ta阻隔层复合线、无阻隔层亚组元设计的Nb3Sn线材,拟采用实验方案及可行性分析:
1) 结构设计及对应工艺路线:线材加工主要采用高Jc和内锡法线材加工工艺路线
a) 新结构Nb3Sn单芯棒采用高Jc Nb3Sn线材的单芯棒,可以根据新结构元素比例等设计要求来具体调整单芯棒Cu/Nb体积比。因此,新结构线材的单芯棒加工工艺和高Jc线材的工艺一致。
b) 新结构Nb3Sn复合棒由采用高Jc Nb3Sn复合棒设计,但高Jc复合棒的Nb筒层被相应的CuNb单芯棒和Cu单芯棒所代替所代替。新结构线材复合棒组装和高Jc线材基本一致,但减少了Nb筒的制备与组装。
c) 新结构Nb3Sn亚组元由采用高Jc Nb3Sn亚组元设计,但高Jc亚组元的Nb阻隔层被相应的CuNb单芯棒和Cu填充。由于非铜超导区Cu元素相比高Jc结构线材的增加,更多的Sn会被多余的Cu消耗,因此亚组元的Nb/Sn原子比应当小于3,使线材处于富Sn结构。新结构线材亚组元制备和高Jc线材工艺一致。
d) 新结构Nb3Sn复合线由Cu包套、Ta阻隔层、以及无阻隔层亚组元单元等共同构成。结构上,这种复合线结构类似于内锡法线材,相比之下,新结构设计复合线具有高Jc Nb3Sn线材的Nb芯丝尺寸(1μm ~ 2μm)和亚组元尺寸(50μm ~ 80μm)、更多的亚组元数量、更小的Cu含量、更高的Nb含量等。性能上,新结构设计线材由于具有高Jc Nb3Sn线材的芯丝结构,它因此相较内锡法线材也会有更高的Ic、Jc等设计性能,但更大体积的非铜超导区则使得新结构Jc会低于高Jc线材的Jc。由于Ta阻隔层的圆形结构,类似于内锡法Nb3Sn复合线的设计方案,新结构同样需要采用三种不同形状的亚组元,分别为六方、圆形和扇形内锡法形状亚组元,新结构复合线加工工艺与内锡法的一致。
e) Nb3Sn线材仍通过对拉伸至最终尺寸的复合线金属元素结构线材进行合适的热处理过程获得。
2) 可行性分析:
a) 结构设计上Nb含量接近高Jc线材结构,理论上可以获得较高性能的线材。
b) 由于新结构中超导区Cu致Sn过多的消耗问题,Sn含量的合理选择需要解决来保证设计性能的获得。
c) 线材可以通过对亚组元的优化很容易获得不同的性能。
d) 加工工艺基本采用目前本单位成熟的加工工艺,线材长线的获得可能更多的与新结构的设计相关。
3) 关键参数验证实验:主要为单参数改变对比实验,实验线材的选择可以是亚组元,或者最终的复合线。
a) 相同结构不同Nb/Sn原子比实验,解决Sn合理选择的问题。
b) 调整CuNb单芯棒Cu/Nb比,获得不同Nb芯丝间距对性能的影响过程,以及Nb芯丝间距的合理范围。
c) 同一设计结构、相同热处理制度不同线径实验,研究不同线径对线材结构设计的关键参数,包括芯丝直径、芯丝间距等。
d) 同一设计结构、不同线径充分热处理反应实验等,研究不同线径线材下热处理制度的调整关键参数。
e) 线材加工时改变加工率,获得加工率对线材变形、断芯等的影响,以及加工率的合理选择范围参数。
f) 研究不同线径下芯丝的变形情况,通过变形分析反馈调整新结构的设计。
g) 其它在研究过程中可能需要补充设计和研究的实验。



特色与创新1000
        1) 本项目致力于获得具有自主结构设计的新型高性能Nb3Sn超导线材,研究具有一定创新开创性,新结构线材设计性能达到2300A/mm2(4.2K,12T)以上,本项目的设计成功将为国内外高磁场应用场景提供新的具有完全自主知识产权的高性能超导线材,进一步提升本单位在高磁场性能超导线材应用领域的技术和产品竞争力。
        2) 本项目研究通过对Nb3Sn超导线材性能关键设计参数的研究,会进一步加深和拓展对Nb3Sn超导线材高性能来源的认知,并为未来更高性能超导线材设计研究提供新的有效的研究思路和方法,本研究项目具有基础机理性研究的特性。
        3) 对长线加工工艺关键参数的研究,旨在现有生产工艺基础上,进一步深入研究不同结构的高性能Nb3Sn超导线材从结构性能设计到工程实现的关键问题,对超导线材加工工艺进一步完善具有很强的现实指导意义。

研究计划与预期成果500
1) 2021.02-2022.01:
第一年度完成方案a,即单Ta阻隔层复合线、无阻隔层亚组元设计的Nb3Sn线材,设计性能大于2000A/mm2,并通过相应加工工艺完成2-3批次新结构线材的加工制备,预期获得对应批次的新结构Nb3Sn线材的加工工艺信息。
完成新结构对应的Nb/Sn比的研究实验,完成线径参数对结构设计关键参数研究实验。获得不同结构线材Nb/Sn比、Cu/Nb比、芯丝间距、芯丝尺寸等关键参数的调整规律。
进行新结构Nb3Sn长线加工工艺的探究,根据之前批次的线材加工信息,研究线材加工率、线材变形等问题,获得新结构线材加工率、变形等对结构调整的影响。
2) 2022.02-2023.01:
设计性能达到2300A/mm2的线材,进行2-3批次的关键参数调整后的新结构制备,获得对应批次Nb3Sn线材。
对2000A/mm2线材进行线材热处理制度方面的实验设计和研究,获得结构调整对应的热处理制度调整策略。
进一步完善性能和工艺加工等关键参数与Nb3Sn线材结构的关系,获得稳定制备高性能新结构超导线材的结构设计规律和长线加工工艺技术。


研究基础( 已取得工作 已具备诶科研条件 尚缺少的科研条件和你解决路径 正在承担的与本项目相关的科研项目情况) 1000
已取得的工作:
        1) 完成第一批次单Ta阻隔层复合线、无阻隔层亚组元结构的Nb3Sn线材的模型设计,设计性能超过1800A/mm2。
2) 本批次新结构线材的Ic、Jc等性能测量结果为设计性能的大约 60\%,目前正在进行这种性能偏差的实验研究。
3) 目前,已经在本单位完成这种结构线材的加工工艺验证和制备,获得百米级别的第一批次新结构线材,表明本单位现有的加工工艺能够满足新结构线材的最终制备。新结构Nb3Sn线材加工时的断线被发现来源于线材中单根Nb芯丝的断裂(单根Nb芯丝断线直径集中在16.6μm ~ 5μm,最大发现 ~ 28μm时单根Nb芯丝的断线),正在进行断线信息的更详细分析和分析工作。

已具备科研条件:
        1) 线材形貌表征,比如金相、SEM等。
        2) 线材性能表征,具有4.2K,12T低维超导性能测量能力等。
        3) 本单位已经具有丰富的超导线材设计和加工数据,可以为本项目研究提供丰富的原始数据的支持。
        4) 本单位具有完整的力学表征手段。
        5) 本单位具有完整的超导线材加工工艺技术。

尚缺少的科研条件:
        1) 本单位线材成分表征技术较为欠缺,联系研究所和大学机构进行所需测量,比如进行EBSD、XRD等。
正在承担的相关项目:
        1) 承担本单位新结构Nb3Sn超导线材制备技术研究项目,为本项目课题研究主要负责人。



\backmatter

\chapter{索引}
\setcounter{secnumdepth}{0} % 不编号


\bibliographystyle{plain}
\bibliography{booktest.bib}

\end{document}